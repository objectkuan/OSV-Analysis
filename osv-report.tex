%
% A analysis report for OSV. 
% 
% Jiaquan He <objectkuan@gmail.com>
% May 11, 2014

\documentclass[10pt]{beamer}
\usetheme{umbc4}
\useinnertheme{umbcboxes}
\setbeamercolor{umbcboxes}{bg=violet!12,fg=black}



\title{OSV Report}
\subtitle{continuous updating}
\author[Jiaquan He, ...]{Jiaquan He, ...}
\institute[THCSOS]{
	THCSOS
}
\date{May 11, 2014}

\begin{document}

%----------- titlepage ----------------------------------------------%
\begin{frame}[plain]
	\titlepage
\end{frame}


%----------- slide --------------------------------------------------%
\begin{frame}
	\frametitle{Overview}

\begin{itemize}
	\item About OSV
	\item Making and Checking OSV
	\item Memory Management
	\item Symmetric Multiprocessing
	\item Scheduling
	\item Synchronization
	\item TCP/IP Protocol Stack
	\item VirtIO Framework
	\item Application Support
\end{itemize}

\end{frame}


%----------- slide --------------------------------------------------%
\begin{frame}
	\frametitle{About OSV}

	OSv is designed from the ground up to execute a single application on top of a hypervisor, resulting in superior performance and effortless management.

	\bigskip
	\pause

	Features: 
	\begin{itemize}
		\item Superior Performance
		\item Rapid VM build and deploy
		\item Zero OS Management
		\item DevOps/PaaS like deployment
		\item Common Java framework integration
		\item Optimize your Native apps
		\item Optimized JVM (coming up)
	\end{itemize}

\end{frame}


%--------------------------------------------------------------------%
%                  Making and Checking OSV
%--------------------------------------------------------------------%
%----------- slide --------------------------------------------------%
\begin{frame}
	\frametitle{Making and Checking OSV}
	\framesubtitle{Making OSV: make all}
	
	The command `make all` makes the OSV image file, which follows the steps:	
	
	\begin{enumerate}
		\item Make \textbf{loader.img}, the bootloader, with scripts/imgedit.py. The loader.elf is contained in loader.img.
		\item Create a 100MB image file \textbf{bare.raw}, and call \textbf{setpartition} in scripts/imgedit.py to write in the content of loader.img.
		\item Convert bare.raw to a qcow2 format file named \textbf{bare.img}, and modify its file size. 
		\item Call scripts/mkzfs.py, which then calls run.py to run OSV, and upload\_manifest.py to upload into OSV all the files listed in \textbf{build/release/bootfs.manifest}. This finishes the file system related jobs.
	\end{enumerate}
\end{frame}


%----------- slide --------------------------------------------------%
\begin{frame}
	\frametitle{Making and Checking OSV}
	\framesubtitle{Checking OSV: make check}
	
	The command `make check` runs the testing cases in OSV.
	
	All testing files are in the \textbf{tests} directory.
	
	\begin{enumerate}
		\item Make the test cases into .o and .so files in build/release/tests/.
		\item Run scripts/test.py, which scan the files in build/release/tests/. Each .so file is a test case. Blacklist is supported to exclude tests. And each case is finished by a calling of scripts/run.py.
		\item The scripts in run.py just choose a certain hypervisor (xen or qemu), to run the image.
		\item Run the tests with the command format `./scripts/run.py -s -e tests/tst-condvar.so`.				
	\end{enumerate}
\end{frame}

%--------------------------------------------------------------------%
%                      Memory Management
%--------------------------------------------------------------------%
%----------- slide --------------------------------------------------%
\begin{frame}
	\frametitle{Memory Management}
	\framesubtitle{Structure}
	
	\textbf{MMU: }
	\begin{itemize}
		\item general: include/osv/mmu-defs.h, include/osv/mmu.h, core/mmu.cc
		\item arch-related: arch/[arch]/arch-mmu.hh, arch/[arch]/mmu.cc
	\end{itemize}
	
	\smallskip
	
	\textbf{Mempool: }
	
	Architecture-independent memory management interfaces.
	
\end{frame}


%----------- slide --------------------------------------------------%
\begin{frame}
	\frametitle{Memory Management}
	\framesubtitle{Page Structure}
	
	\center
	\texttt{
	|<----------- page size ---------->|
	|------|-----------|---|-----------|
	|Header|free\_object|...|free\_object|
	|------|-----------|---|-----------|
	}
	
\medskip

	Header holds the information: CPU id, the owner mempool, allocated amount, and the free objects.
	
\end{frame}



%--------------------------------------------------------------------%
%                      Symmetric Multiprocessing
%--------------------------------------------------------------------%
%----------- slide --------------------------------------------------%
\begin{frame}
	\frametitle{Symmetric Multiprocessing}
  
	
\end{frame}



%--------------------------------------------------------------------%
%                           Scheduling
%--------------------------------------------------------------------%
%----------- slide --------------------------------------------------%
\begin{frame}
	\frametitle{Scheduling}


\end{frame}



%--------------------------------------------------------------------%
%                           Synchronization
%--------------------------------------------------------------------%
%----------- slide --------------------------------------------------%
\begin{frame}
	\frametitle{Synchronization}
	\begin{itemize}
		\item lfmutex
		\item condvar
		\item preempt-lock
		\item rcu
		\item rwlock
		\item semphore
		\item spinlock
	\end{itemize}
\end{frame}

%----------- slide --------------------------------------------------%
\begin{frame}
	\frametitle{Synchronization}
	\framesubtitle{lfmutex - Lock-free Mutex}
	
	\textbf{Algorithm} \\
	Responsibility Handoff Protocol - Gidenstam A, Papatriantafilou M. LFthreads: A lock-free thread library[M]//Principles of Distributed Systems. Springer Berlin Heidelberg, 2007: 217-231.

	\smallskip
	
	\textbf{Regular Functions}
	\begin{itemize}
		\item \texttt{void lock()}
		\item \texttt{void unlock()}
		\item \texttt{void try\_lock()}
		\item \texttt{bool owned()}
	\end{itemize}
	
	\smallskip
	
	\textbf{Wait Morphing Functions}
	\begin{itemize}
		\item \texttt{void send\_lock(wait\_record *wr);}
		\item \texttt{bool send\_lock\_unless\_already\_waiting(wait\_record *wr);}
		\item \texttt{void receive\_lock();}
	\end{itemize}
	
\end{frame}

%----------- slide --------------------------------------------------%
\begin{frame}
	\frametitle{Synchronization}
	\framesubtitle{condvar - Conditional Variable}

	\textbf{Theories}
	Using the morphing functions to implement.
	
	\smallskip

	\textbf{Functions}
	\begin{itemize}
		\item \texttt{int wait(mutex *user\_mutex, std::chrono::time\_point<Clock, Duration> time)}
		\item \texttt{int wait(mutex *user\_mutex, std::chrono::duration<Rep, Period> duration);}
		\item \texttt{int wait(mutex* user\_mutex, sched::timer *tmr = nullptr);}
		\item \texttt{void wait\_until(mutex\& mtx, Pred pred);}
		\item \texttt{void wake\_one();}
		\item \texttt{void wake\_all();}
	\end{itemize}
	
\end{frame}

%----------- slide --------------------------------------------------%
\begin{frame}
	\frametitle{Synchronization}
	\framesubtitle{preempt-lock - Enable/Disble Preempt}

	\textbf{Theories}
	Using a preempt counter to control the preemption.
	
	\smallskip

	\textbf{Functions}
	\begin{itemize}
		\item \texttt{void lock();}
		\item \texttt{void unlock();}
	\end{itemize}
	
\end{frame}
%percpu%

%--------------------------------------------------------------------%
%                         TCP/IP Protocol Stack
%--------------------------------------------------------------------%
%----------- slide --------------------------------------------------%
\begin{frame}
	\frametitle{TCP/IP Protocol Stack}

\end{frame}

\begin{frame}
	\frametitle{VirtIO}

\end{frame}


%--------------------------------------------------------------------%
%                         Application Support
%--------------------------------------------------------------------%
%----------- slide --------------------------------------------------%
\begin{frame}
	\frametitle{Application Support}
	

\end{frame}



%--------------------------------------------------------------------%
%                        Post Report
%--------------------------------------------------------------------%
%----------- slide --------------------------------------------------%
\begin{frame}
	\frametitle{OSV Report}

\center
Thanks

\end{frame}

\end{document}
