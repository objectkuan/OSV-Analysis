%
% A analysis report for OSV. 
% 
% Jiaquan He <objectkuan@gmail.com>
% May 11, 2014

\documentclass[10pt]{beamer}
\usetheme{umbc4}
\useinnertheme{umbcboxes}
\setbeamercolor{umbcboxes}{bg=violet!12,fg=black}


\title{OSV Report}
\subtitle{continuous updating}
\author[Jiaquan He, ...]{Jiaquan He, ...}
\institute[THCSOS]{
	THCSOS
}
\date{May 11, 2014}


\begin{document}

%----------- titlepage ----------------------------------------------%
\begin{frame}[plain]
	\titlepage
\end{frame}


%----------- slide --------------------------------------------------%
\begin{frame}
	\frametitle{Overview}

\begin{itemize}
	\item About OSV
	\item Memory Management
	\item Symmetric Multiprocessing
	\item Scheduling
	\item Synchronization
	\item TCP/IP Protocol Stack
	\item VirtIO Framework
	\item Application Support
\end{itemize}

\end{frame}


%----------- slide --------------------------------------------------%
\begin{frame}
	\frametitle{About OSV}

	OSv is designed from the ground up to execute a single application on top of a hypervisor, resulting in superior performance and effortless management.

	\bigskip
	\pause

	Features: 
	\begin{itemize}
		\item Superior Performance
		\item Rapid VM build and deploy
		\item Zero OS Management
		\item DevOps/PaaS like deployment
		\item Common Java framework integration
		\item Optimize your Native apps
		\item Optimized JVM (coming up)
	\end{itemize}

\end{frame}

%----------- slide --------------------------------------------------%
\begin{frame}
	\frametitle{Memory Management}
  
\end{frame}


%----------- slide --------------------------------------------------%
\begin{frame}
	\frametitle{Symmetric Multiprocessing}
  
	
\end{frame}



%----------- slide --------------------------------------------------%
\begin{frame}
	\frametitle{Scheduling}


\end{frame}


%----------- slide --------------------------------------------------%
\begin{frame}
	\frametitle{Perf Stat}

For any of the supported events, perf can keep a running count during process execution. In counting modes, the occurrences of events are simply aggregated and presented on standard output at the end of an application run. To generate these statistics, use the stat command of perf.

\end{frame}


%----------- slide --------------------------------------------------%
\begin{frame}
	\frametitle{Synchronization}

\end{frame}


%----------- slide --------------------------------------------------%
\begin{frame}
	\frametitle{TCP/IP Protocol Stack}

\end{frame}

\begin{frame}
	\frametitle{VirtIO}

\end{frame}


%----------- slide --------------------------------------------------%
\begin{frame}
	\frametitle{Application Support}
	

\end{frame}



%----------- slide --------------------------------------------------%
\begin{frame}
	\frametitle{Perf Report}

\center
Thanks

\end{frame}

\end{document}
